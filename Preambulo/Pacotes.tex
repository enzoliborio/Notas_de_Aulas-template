\usepackage[utf8]{inputenc}	% Codificacao do documento (conversão automática dos acentos)
\usepackage[T1]{fontenc}		% Selecao de codigos de fonte.
\usepackage{lmodern}	        % Usa a fonte Latin Modern


%-----------------------------------------------------------
% BIBLIOGRAFIA
%-----------------------------------------------------------
% \usepackage[style=abnt, backref=false, language=brazil]{biblatex} % Carrega o estilo ABNT
\usepackage{biblatex}
\usepackage[brazilian]{babel} % Suporte ao idioma brasileiro
\addbibresource{Referencias.bib}


% ---------------------------------------
% PACOTES ESSENCIAIS
% ---------------------------------------
% CORES
\usepackage[table,xcdraw,dvipsnames]{xcolor}
\definecolor{mintedgreen}{HTML}{155C12}
\definecolor{mintedgreenNovo}{HTML}{0B610B}

\usepackage{graphicx} % imagens
\usepackage{float} % para \begin{figure}[H]
% \usepackage{caption} % legenda de figuras\tabelas
\usepackage{subcaption} % para usar o ambiente subfigure/subcaption

\captionsetup[figure]{font={footnotesize}} % ALTERAÇÃO: fonte menor na legenda de figuras
\captionsetup[table]{font={footnotesize}} % ALTERAÇÃO: fonte menor na legenda de tabelas
% \usepackage{subfigure} %para subfiguras (Figura 1a, 1b, ...)

\usepackage{booktabs} % Para \toprule, \midrule, \bottomrule

\usepackage{pdfpages} % CAPA
\pdfmajorversion=1
\pdfminorversion=7

\usepackage{emptypage} % Garante que páginas em branco sejam realmente em branco
\usepackage{appendix} % Apêndice

\usepackage{indentfirst} % Indenta o primeiro parágrafo
\usepackage{microtype} % para melhorias de justificação
\usepackage{setspace} % para \begin{spacing} [...] \end{spacing}

\usepackage{enumerate} % listas enumeradas
\usepackage{enumitem} % listas pontilhadas

\usepackage{titlesec} % Personalização de títulos
% O tamanho do parágrafo é dado por:
\setlength{\parindent}{1.3cm}
% Controle do espaçamento entre um parágrafo e outro:
\setlength{\parskip}{0.2cm}

% escita matemática:
\usepackage{amsthm}
\usepackage{amsmath}
\usepackage{amssymb,amsfonts,amsthm}
\usepackage{bigints} % integrais maiores

\usepackage{siunitx} % para \celsius

\usepackage{url} % usar \url{}
\urlstyle{same}

% \usepackage{chngcntr} % Pacote necessário para redefinir os contadores
% \counterwithin{figure}{section} % Faz as figuras serem numeradas por subsection

% \usepackage{extsizes} % Suporte para tamanhos de fonte maiores

\usepackage{lipsum} % para texto de exemplo

\usepackage{graphics}


% ---------------------------------------
% AMBIENTE \begin{blocks}
% ---------------------------------------
\usepackage[most,theorems]{tcolorbox}

\newtcolorbox{blocks}[1]{
	colback=black!5!white,     % Cor de fundo do corpo do texto
	colframe=black!70!white,   % Cor da fundo do título
	fonttitle=\bfseries,
	title={#1},
	enhanced,                  % Melhora a aparência
	boxrule=0.4pt,             % Espessura da borda
	arc=4pt                    % Arredondamento dos cantos
}

\newtcbtheorem[auto counter, number within=chapter]{teorema}{Teorema}{%
    colback=mintedgreen!5!white,    % Cor de fundo do corpo do texto
	colframe=mintedgreen!70!white,  % Cor da fundo do título
	fonttitle=\bfseries,
    title={#1},
	enhanced,                      % Melhora a aparência
	boxrule=0.4pt,                 % Espessura da borda
	arc=4pt,                       % Arredondamento dos cantos
}{teorema} % Prefixo para labels (ex: \label{teorema:meu_teorema})

\theoremstyle{definition} % Texto normal, título em negrito
\newtheorem{exemplo}{\( \blacksquare \) Exemplo}[chapter] % para exemplos numerados pelo capítulo





% ---------------------------------------
% HYPER LINKS
% ---------------------------------------
\usepackage{hyperref}
\hypersetup{
	%pagebackref=true,
	colorlinks=true,       		% false: boxed links; true: colored links
	linkcolor=black,          	% color of internal links
	citecolor=OliveGreen,            % color of links to bibliography
	filecolor=magenta,      	% color of file links
	urlcolor=OliveGreen,
	bookmarksdepth=4
}

% ---------------------------------------
% GRÁFICOS
% ---------------------------------------
\usepackage{mathrsfs}
\usetikzlibrary{arrows}
\usepackage{tikz}
\usepackage{pgfplots}
\usetikzlibrary{
	angles, % Para desenhar o ângulo theta
	quotes, % Para colocar o rótulo no ângulo
	decorations.pathreplacing % Para desenhar a chave sob o eixo x
}

%usepackage[mode=buildnew]{standalone}
%\usepackage{import}
\pgfplotsset{compat=1.9}
%\usepgfplotslibrary{patchplots}
%\usepgfplotslibrary{external}
%\usetikzlibrary{external}
%\tikzset{
	%external/system call={xelatex \tikzexternalcheckshellescape -halt-on-error -interaction=batchmode -jobname "\image" "\texsource"}}
%\tikzexternalize[prefix=tikz/]


% ---------------------------------------
% PARA CÓDIGOS
% ---------------------------------------
\usepackage{fvextra} % Necessário para 'linenos' funcionar
\usepackage{csquotes} % adicionar DEPOIS de \usepackage{fvextra}
\MakeOuterQuote{"} % para usar aspas "

\usepackage{minted}
\setminted{
    %style=friendly,     % 4. Muda o estilo para um sem itálico nos comentários
    %fontsize=\small,
    linenos,            % Habilita números de linha
    breaklines,
    frame=single,
    xleftmargin=1em,    % 1. Empurra tudo para dentro da margem
    numbersep=5pt       % 2. Aproxima os números do código
}

% \usepackage{listings}
% \renewcommand{\listingscaption}{Código}

\usepackage{newfloat}
\DeclareFloatingEnvironment[fileext=loc, within=chapter]{codigo}
\floatname{codigo}{Código} % nome na legenda
\captionsetup[codigo]{skip=-3mm} % distância entre código e legenda
